\chapter{Introduction}
\label{chap:introduction}
The environment in which we are immersed every day is pervaded by devices capable of computing and communicating.
This situation leads to design smart-environment like smart-city, smart-home, smart-hospital, and so on.
All these systems, and more in general, the cyber-physical systems (CPS), are composed of myriads of heterogeneous devices.
The devices can differ in their computational and communication capabilities, and ability to interact with the environment in which they are immersed.
This heterogeneity makes difficult to design open distributed and technology independent systems, particularly with regard to communication technology.
In fact exists several wireless communication technologies that differ for their potentiality and limits, like transmission range, energy consumption, cost, and so on.
Among all the technologies, one of the most promising is \mbox{LoRaWAN}~\cite{loraalliancetechnicalcommittee2020}, which allows transmission with low power consumption and long range.

To make easier design these kind of systems in literature are presents several approaches.
One class of approaches proposes to define a global platform. 
It should provide a uniform communication interface that allows the application to abstract from the real communication technology used by the devices. 
One example is Sentilo\footnote{\href{http://www.sentilo.io}{http://www.sentilo.io}}, a cross platform middleware designed for the smart-city of Barcelona.
It provides a simple REST interface to send and receive sensor data, and a set of core modules to govern the system like real time storage, and network security. 
Its architecture is extendible and allows horizontal scalability from single servers to cluster.
% 
Another approach is the aggregate computing paradigm~\cite{BealIEEEComputer2015}.
It increases the level of abstraction respect the previous one, allowing to define programs where the programmable entity is not the single entity but the all set of devices.

The behaviour of a complex pervasive system, and moreover of an aggregate system, cannot be tested on a single machine; at the same time, testing in the real world is usually inconvenient, complex, and expensive, or even straightforwardly impossible.
A partial solution to test the behaviour of a system prior to deployment is via simulation.
In literature exists several simulator for different domains and with different models to allow simulations at different abstraction levels.
One simulation tool, designed in University of Bologna, is Alchemist~\cite{PianiniJOS2013}.
Alchemist has been used in the past, for instance, for simulating: crowd detection~\cite{FGCS2017, TOMACS2018} and evacuation~\cite{DBLP:journals/corr/abs-1711-08297}, for mixed edge-cloud computation~\cite{Coordination2019-sgcg}, opportunistic instant messaging~\cite{Coordination2019-processes}, smart vehicle counting~\cite{Viroli2016}, etc.
So Alchemist allows also simulation of aggregate systems, but it actually does not allow to simulate the behaviour of a communication network, like a LoRaWAN one, in a realistic way.
Another relevant platform, but specialized to simulate the behaviour of a LoRaWAN network on large scale is the DingNet Simulator~\cite{inproceedings}.
This platform is useful to simulate CPS that includes a high number of Long Range (LoRa)-based devices; considering also the high-density of devices that can be integrated in these networks~\cite{Lavric2019}. 

Considering all, the purpose of this thesis is to start filling the network abstraction gap of the aggregate paradigm beginning with supporting aggregate computing over LoRaWAN networks.
To do so two main activities will be performed: 
\begin{itemize}
    \item design the support for Protelis, which is a language for aggregate computing, over LoRaWAN
    \item provide a simulation platform for this kind of systems,
    that should also allow to deploy the simulated systems in real networks with a low effort.
\end{itemize}
To obtain the desired simulation platform, choosing DingNet as baseline, will be necessary to improve and evolve it adding all the needed functionalities. 

%
\paragraph{Thesis Structure.} % Optional paragraph title
%%
Accordingly, the reminder of this thesis is structured as follows.
%
\Cref{chap:background} provides an overview of the aggregate computing paradigm, \mbox{LoRaWAN} protocol, and the DingNet simulator.
%
\Cref{chap:contribution} presents the work done to improve DingNet and evolve it in a platform that supports aggregate computing applications.
%
\Cref{chap:case-staudyLoRa} presents a case study developed on DingNet simulator after its evolution for the Day of Science in Flanders.
% 
\Cref{chap:case-staudyAC} presents a case study where an aggregate computing application is simulated in the DingNet simulator.
% 
Finally, \Cref{chap:conclusions} concludes this thesis by summarising its main contribution, and presenting future works.