\chapter{Contribution}
\label{chap:contribution}
This chapter explain the contribution of this thesis. \Cref{sec:contributionDingNet} present all the work done to extend and evolve the DingNet simulator\todo{demo??????}. \Cref{sec:contributionACOverDingNet} present the work done to support the execution of aggregate computing program over the DingNet simulator, and in particular Protelis program. 

\section{Extension and evolution to DingNet}
\label{sec:contributionDingNet}
% code cleanup (use type and base interface when possible, reduce code duplication, improve legibility of code (reduce magic number, useless function, ...)), define unit of measure, conversion of the map environment from discreet to continuous one (int position to double and geo, mote movement from manathan style to normal, ...), implement concept of MQTT broker, introduce concept of network-server, improve exentdibility of mote for how manage incomining message and for communication protocol mote-gateway, add concept of userMote, redesing simulation loop using timer event instead loop with time checking, change build tool to gradleKts, export transmission statistic outside the mote, lavoro per la demo?????, ranged sensor, concetto del tempo da LocalTime a DoubleTime

\section{Aggregate programming over DingNet}
\label{sec:contributionACOverDingNet}
% AC con round poco frequenti?
% possibili mapping di dove computiamo chi (cit pianini perchè no direttamente su LoRa)

\subsection{Protelis-DingNet integration}

% unica parte fondamentale execContext che si sottoscrive a topic mqtt del rispettivomote e aggiorna env con nuove rilevazioni dei sensori e posizione del mote se sensore gps.
% Il resto è tutto un di più (nodi che implementano interfaccie per posizione, network-manager basato su mqtt, vicinato basato su distanza linea d'aria, aggiornamento vicinato ad ogni combiamento di posizione, esecuzione VMrun con timer simulatore