\chapter{Case studies}
\label{chap:case-studies}
This chapter illustrates two case studies developed on the DingNet simulator.

The case study in \cref{sec:case-staudyLoRa} shows a the DingNet platform after its extension.
The main new features used are the application layer that enables the communication between LoRaWAN gateways and applications, the managing of the incoming messages mote-side, and the new type of sensor.  
To do so, a system that requires a bi-directional communication between the application and the motes, and with the application behaviour dependent on the payload of the motes packets is conceived.

The case study in \cref{sec:case-staudyAC} exploits all the new features added in the DingNet simulator.
It uses the integration for the execution of Protelis application over DingNet, and the new type of simulation (\textit{TimedRun}) in addition to the features already used in the previous case study.
% unire i due capitoli
% avere due macro sezioni -> uno per caso
% farlo vedere come lavoro progressivo
% come prima cosa si vuole mettere in opera la piattaforma dingnet estesa con i nuovi requisiti ecc, per esercitarli si è pensato ad un caso di studio pensato in questa maniera con blabla 
% il secondo case study invece esercita la piattaforma alla sua piena potenza quindi sfrutta il già testato ambaradm di dingnet esteso e la nuova integrazione di protelis su dingnet

\chapter{Case study: Navigation in a LoRaWAN network}
\label{chap:case-staudyLoRa}

This chapter presents the case study used as reference scenario during the evolution of the DingNet simulator, and to evaluate the limits of communication from applications to LoRa motes. 
A demo from this case study was showed during the Day of Science in Flanders to present the LoRaWAN technology receiving a lot of attention.

\section{Description case study}
Leuven is cycling city where most of the inhabitants and students use daily their bike to move across the city. 
In this context we want to realize a system able to provide to the user the healthiest route to reach a destination. 
The route generation will be based on air quality level of the areas to across to reach the destination. 
The user, in order to obtain the route, has to require it to the application deployed in a remote server.
The system use the data received from the sensor network to create a city map of quality air.
Then, the application has to define the route to a destination that optimize the trade-off between air quality and length of the route.
The system should be able to recompute the best route if the environment condition change, and communicate it to the user.
The sensor network can be composed by two types of sensors:
\begin{itemize}
    \item \textbf{Fixed}: positioned along the roads and at intersections
    \item \textbf{Mobile}: placed on public transport or bicycles
\end{itemize}
All the sensors have to be deployed in a LoRaWan network, similarly also the user device, that will interact with the application to require the route, has to use the LoRa technology.

\section{Design of the system}
\autoref{fig:caseStudyA} shows the high level architecture of the system, and introduce the main entities. 
The main entities are the sensor devices, the user devices, and the routing application.
As requirement both the sensor devices and the user devices have to be displaced inside the LoRaWAN network and use the LoRa technology to communicate with the routing application.
Using the DingNet simulator to simulate the LoRaWAN network:
\begin{itemize}
    \item the routing application is mapped in a generic application presents in the application server that communicates with the LoRaWAN network via MQTT
    \item the sensor devices, that have to send only packet with the sensed value, can be mapped in the \textit{Mote} simulator entity
    \item the user devices are special devices, because they do not send only packets with the sensed value, but have also to require the route for a destination and be able to manage the received packets with the route. Actually there isn't present a simulator entity with this specific abilities, so it will be necessary to define it.
\end{itemize}
% 
\begin{figure}[h]
    \centering
    \includegraphics{figures/CaseStudyA_HLarch.png}
    \caption{High level architecture of the system}
    \label{fig:caseStudyA}
\end{figure}
% 
\subsection*{Interaction between application and devices}
If on one hand the transmissions from the devices (both sensor and user) to the application are in compliance with the maximum packet's length defined by the LoRaWAN standard (1 byte for packets with the sensed value, and 16 bytes for the packet to require the route), on the other hand it is impossible to send the packet from application to the user device with the entire route, so the only way to send it is to split it in more packets.
In order to send the entire route to the user device are possible two approaches: 
\begin{enumerate}
    \item define a specific interaction protocol to send all the packets with the entire route to the user device immediately after its computation
    \item send a packet with a part of the route only when the user have finish the previous part of the route.
\end{enumerate}
Considering also the requirement to recompute the route if the environment condition change it was chosen the second option, enabling to recompute the route before send its next part.

\subsection*{Design of the user device}
% UserMote - consume packet
The user device has to perform three activity: require the route, send update of its position, and manage the incoming packets.
If on one hand the last two activity can be performed also from a \textit{Mote}, on the other hand it cannot require the route.
For this reason in \autoref{fig:userMote} is introduced a new entity simulator, the \textit{UserMote} that extends the \textit{Mote} adding the logic to require the route when needed sending a packet with starting and destination positions.
Then to manage incoming packets are chosen \textit{MaintainLastPacket} as strategy to store packets until that they are consumed, and \textit{ReplacePath} as only strategy to consume packets. It consume each packet updating the user route in accord with the packet content until the route is completed.
To perform the last activity, send updating of the user position, is necessary add the GPS sensor to the \textit{UserMote}, and send it when the user is closer to the destination of sub-route.
% 
\begin{figure}[h]
    \centering
    \includegraphics{figures/userMote.png}
    \caption{\textit{UserMote} model.}
    \label{fig:userMote}
\end{figure}
% 

\subsection*{Routing application}
The application to found the best route implements an A-star algorithm on the graph of street of the city. 
The weight of the edges of the graph corresponds to the distance between the two points multiply for a factor that represent the air quality level in that street. 
The values sensed by the sensors are retrieved subscribing the topics where the LoRaWAN networks publish them, and the route is send to the user mote publishing the massage with sub-route in its receiving topic.
The application recompute the best route only when the user device communicate its new position and if some environment condition is changed from the previous computation.

\section{Simulation in DingNet}
% Demo + discussion limitazioni ed utilità del caso di studio????
We here present simulations in small scale conducted over the the city of Leuven with the DingNet simulator. 
The environment is composed from 2 gateways, 4 fixed sensors, one mobile sensor that follow a path like a public transport, and a user that require a route for a destination.
All the types of LoRa devices are configured in a way to try to reduce collision between transmissions.
\autoref{fig:sim1} shows three snapshots of a simulation run, where the transparent layer represents the air quality level in that point based on the received sensor value.
First snapshot shows the first computed route in blue, and in red the sub-route received from the user until that moment.
Second snapshot shows how after an environment conditions change, the route is recomputed with a longer one, but considered better from the application combining distance and pollution. 
So the user has received a new sub-route of the new best route that start from its new position.
Finally, the last snapshot shows the user arrived to destination.
% 
\begin{figure}[h]
    \centering
    \begin{tabular}{lll}
         \includegraphics[scale=0.42]{figures/sim1snap1.png}  &
         \includegraphics[scale=0.42]{figures/sim1snap2.png} &
         \includegraphics[scale=0.42]{figures/sim1snap3.png} 
    \end{tabular}
    \caption{Three snapshot of a simulation run with changing of the route due to the change of the environment conditions}
    \label{fig:sim1}
\end{figure}
% 

\noindent \autoref{fig:sim2} shows a different user that require a path for the same destination. In this case its best route never changes because the changes of the environment condition do not affect the areas across from its route.
% 
\begin{figure}[h]
    \centering
    \begin{tabular}{lll}
         \includegraphics[scale=0.42]{figures/sim2snap1.png}  &
         \includegraphics[scale=0.42]{figures/sim2snap2.png} &
         \includegraphics[scale=0.42]{figures/sim2snap3.png} 
    \end{tabular}
    \caption{Three snapshot of a simulation run without changing of the route despite the change of the environment conditions}
    \label{fig:sim2}
\end{figure}
% 

Although it is a simple simulation with few sensors and only one user device, it can be deduced that use LoRa technology also for user devices is possible only in certain scenarios.
In this scenario the number of transmissions necessary to transmit all the route is high, so according to LoRaWAN limitations it is possible only for short route.
Moreover considering a real scenario where the number of user is higher (es. more the one thousand) the network congestion will increase with also the probability of collision among transmissions, which require re-transmission worsening the situation.
\paragraph{Concluding} this case study has been useful as reference for the simulator evolution helping to consider the behaviour of all the network facilities and the bi-directional communication schema.
It has confirmed the validity of LoRaWAN as enabling technology for a sensor network and its limits regarding the communication from application to LoRa devices.

\iffalse
Seppur è una simulazione semplice con pochi sensori e un solo dispositivo utente per volta, si può dedurre che utilizzare la tecnologia LoRa anche per i dispositivi utente è una scelta attuabile solo in alcuni scenari. Infatti il numero di trasmissioni necessario per comunicare l'intero percorso è elevato e quindi possibile solo per brevi tragitti. 
Inoltre considerando la presenza di più dispositivi utenti (es 100+) la congestione della rete aumenterebbe, ma anche la probabilità di collisioni tra le varie trasmissioni, andando a richiedere successive ritrasmissioni peggiorando quindi la situazione.

Concludendo questo caso di studio è stato utile come riferimento per l'evoluzione del simulatore nel considerareil comportamento di tutti i dispositivi di rete e la comunicazione in entrambe le direzioni. 
Ha confermato la validità di LoRaWAN come tecnologia abilitante per una rete di sensori e i suoi limiti per quanto riguarda la comunicazione da applicazione a dispositivi LoRa.
\fi
\chapter{Case study: Central heating and pollution level}
\label{chap:case-staudyAC}
This chapter presents a case study that joins the aggregate computing paradigm and a LoRaWAN network. Finally, it show a simulation of the case study in the DingNet simulator.

\section{Description case study}
Nowadays air pollution is a very common problem of cities of all the world.
Two of the main strategies used to reduce the emission of polluting gas are traffic bans in strategic city's areas and maximum temperature allowed in public and private building heating.

In this context we want to realize a monitoring system for the air quality based on the CAQI index~\cite{CAQI}, which is able to apply strategies to maintain under control the air pollution level.
The sensor network is composed of a set of fixed sensors scattered around the city, and mobile sensors placed on public transports (like bus or public bicycles).
All the sensor devices are equipped at least with a sensor for the particular matter 10 (PM10).
The idea is to displace strategically the fixed sensors to achieve a good city coverage, using mobile sensors for its refinement and to reduce reading errors of the fixed ones.

In a first step to reduce air pollution it has been chosen to control the maximum temperature allowed for the heating of buildings.
The idea is to allow the system to manage the building heating systems control devices (from now on building devices). So the system can set the maximum reachable temperature based on pollution level of its area.

All the sensors have to be displaced in a LoRaWAN network to communicate their sensed data to reduce their cost and the cost for their maintenance. 
Building devices have not any particular requirements, they have not problems of energy consumption because they can be connected to the power line of the building.
The same policy can be adopted for the Internet connection, this allows to avoid to use LoRa technology reducing the number of LoRa devices in the network and increasing their communication capability.

\section{Design of the system}

Aggregate computing is a good approach for this system for several reasons.
First of all, it is a heterogeneous system composed for at least two types of devices (sensor and building device) with different capabilities like connectivity, computational resources, and their interaction with the environment. 
Furthermore it is composed from an high number of devices (one device building for each house of the city, a set of fixed sensors, and a set of mobile ones) so scalability can be a problem. But with aggregate computing is possible to solve it scaling horizontally and moving the computational node in different network devices without the need to change the program.
\autoref{fig:caseStudyAC} shows a high level architecture of the system.
\begin{figure}[h]
    \centering
    \includegraphics{figures/caseStudyB_high.png}
    \caption{High level architecture of the system}
    \label{fig:caseStudyAC}
\end{figure}
\\Designing the system with the aggregate computing is important to model the two different kind of entities and the communication network for the aggregate nodes.

\subsection*{Entities model}
The sensor devices are LoRa motes, so they are mapped in \mbox{\textit{Mote}} inside DingNet, while, according to \cref{sec:PoverD}, in the aggregate application they can be mapped in simple \mbox{\textit{ProtelisLoRaNode}}.
The building devices are not LoRa motes, so they do not require to be mapped in \mbox{\textit{ProtelisLoRaNode}} and they can be generic Protelis node.
Anyway it is decided to map them in \mbox{\textit{BuildingNode}}, which extends \mbox{\textit{ProtelisLoRaNode}}, \autoref{fig:caseBmodel_a}. This because:
\begin{itemize}
    \item the building node has not a physical mote then its topic will never receive a MQTT message; so it has not any overhead
    \item all the types of nodes have a base \mbox{\textit{ExecutionContext}} where introduce functions domain specific
    \item if in the future they will have a physical mote, it will not be necessary to change the system architecture.
\end{itemize}
\autoref{fig:caseBmodel_b} completes the model of the entities with their \mbox{\textit{ExecutionContext}}.
\mbox{\textit{SensorExecutionContext}} updates the knowledge-base of the node computing the CAQI index at each new sensed data received. It contains also all the methods for the logic domain specific. \mbox{\textit{BuildingExecutionContext}} extends it adding the capability to modify the temperature in its physical counterpart.
\begin{figure}[h]
    \centering
    \begin{subfigure}{.495\textwidth}
        \centering
        \includegraphics{figures/NodeAC_caseStudy.png}
        \caption{}
        \label{fig:caseBmodel_a}
    \end{subfigure}
    \begin{subfigure}{.495\textwidth}
        \centering
        \includegraphics{figures/ECAC_caseStudy.png}
        \caption{}
        \label{fig:caseBmodel_b}
    \end{subfigure}
    \caption{Model of the aggregate entities.}
    \label{fig:caseBmodel}
\end{figure}
\subsection*{Interaction between Protelis nodes}
To complete the backend required from a Protelis application we need to:
\begin{enumerate}
    \item design and implement the communication between the Protelis node
    \item define the neighbourhood policy to define the neighbours of each node
\end{enumerate}
As neighbourhood policy it has been chosen a distance based policy. 
This policy has been chosen considering the application domain, in fact the behaviour of each node depends on the environment state in its area. 

\noindent To implement the \mbox{\textit{NetworkManager}} and enable the communication between Protelis nodes it has been chosen to use MQTT.
MQTT has been chosen because it is a lightweight protocol and enable devices to send the same message to more devices with only one communication.
This is very important in a large scale system where the nodes are displaced in many places and the connectivity can go down.

\noindent This system is composed also from mobile nodes and when one node change position its neighbours can change, as the neighbourhood of other nodes. 
So the definition of the neighbourhood cannot be done only at configuration time, but also after.
To do that it is defined the \mbox{\textit{NeighborhoodManager}}, which manage the neighbourhood of all the nodes.
It receives the update of the node position, recomputes the neighbourhoods, and communicates their to all the nodes. 
This entity allows also to modify the composition of the system at runtime adding or removing entities with all the neighbourhood updated automatically.

\section{Protelis program}
After presenting the design of the Protelis back-end, this section presents the Protelis program for the global behaviour of the system. 
The program, visible in \autoref{lst:program}, requires only 25 lines of code.
Methods \mbox{\textit{decreaseTemp}} and \mbox{\textit{increaseTemp}} modify the temperature of the building device of a delta temperature every half an hour.
These methods are build on top of the function \mbox{\textit{cyclicFunction}}, which is present in the developer API of the aggregate stack.
Lines 16 - 23 first create a computational field of sensed values and distances from the respective sensor. Then manipulate it to define a field of maximum temperature allowed for each device based on CAQI index.
The final part of the program defines the target temperature for each device and selects the correct method to achieve it. 

\lstinputlisting[
	float,
	language=Protelis,
	caption={Protelis program for monitoring application},
	label={lst:program},
]{listings/homeHeating_timer.pt}

\section{Simulation in DingNet}

Here is presented a simulation conducted over the city of Leuven with the DingNet simulator. 
The environment is composed from the following entities:
\begin{itemize}
    \item 9 of the 11 DingNet network gateways (the others two are outside of the simulation region)
    \item 8 fixed sensors equipped with the PM10 sensor
    \item 2 mobile sensors equipped with PM10 and GPS sensors
    \item 3 building devices displaced in three different areas of the city.
\end{itemize}
The low number of mobile sensors and building devices is only due to visualisation reasons. 
All the sensors are configured to send a new measurement every hour, according to the specifications of the CAQI index for this polluting gas.
\autoref{fig:simAC} shows three snapshots taken from a simulation run of five days.
The transparent layer represents the air quality level, that is obtained applying an inverse distance weighting on sensed values in the range of 1Km. 
Green color means ``very low'' level, while red color means ``high'' level.
The building devices are represented with a black dot and a text with pattern ``X/Y/Z''.
% 
\begin{figure}[h]
    \centering
    \begin{tabular}{ll}
         \includegraphics[scale=0.42]{figures/simACsnap1s.png}  &
         \includegraphics[scale=0.42]{figures/simACsnap2s.png}
    \end{tabular}
    \begin{tabular}{c}
         \includegraphics[scale=0.42]{figures/simACsnap3s.png} 
    \end{tabular}
    \caption{Three snapshot of a simulation run.}
    \label{fig:simAC}
\end{figure}
% 
X is its current temperature, Y is its desired temperature, and Z is its maximum reachable temperature based on pollution level.
The three snapshots show how the pollution changed during the five days of simulation, and how the building maximum reachable temperature is adapted consequently.
In particular the building in the north of the city has always the desired temperature greater then the maximum reachable. 
The building in the center of the city has first the desired temperature greater then the maximum reachable, but after the pollution change it is the opposite and the building reach its desired temperature.
Finally, the building in the south has always the desired temperature lower then the maximum reachable.

\paragraph{Concluding} this chapter has presented a system designed with the aggregate computing paradigm over a LoRaWAN network, and simulated in the DingNet simulator.
After all the improvements and extensions on DingNet simulator, this case study has showed how it is possible simulate Protelis simulation in this platform. 



\paragraph{Concluding remarks.} This chapter illustrated two case-studies developed on the DingNet simulator.
The first one showed the new application layer of the simulator and the new main features. It confirmed the validity of LoRaWAN as enabling technology for a sensor network and its limits regarding the communication from application to LoRa devices.
The second case study showed an application developed using all the features of the platform and proved its validity as platform to simulate Protelis applications.

% This chapter has presented the case study used as reference for the simulator evolution. It has helped to consider the behaviour of all the network facilities and the bi-directional communication schema. It has confirmed the validity of LoRaWAN as enabling technology for a sensor network and its limits regarding the communication from application to LoRa devices.

% \paragraph{Concluding.} This chapter has presented a system designed with the aggregate computing paradigm over a LoRaWAN network, and simulated it in the DingNet simulator. After all the improvements and extensions on DingNet simulator, this case study has showed how it is possible simulate Protelis applications in this platform.