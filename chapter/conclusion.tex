\chapter{Wrap-up}
\label{chap:conclusions}
\section{Conclusion} 
This thesis focus on providing a platform to simulate aggregate systems over \mbox{LoRaWAN} networks.
At first, the focus has been on the improvement and the extension of the DingNet simulator, by refining its model and adding new features.
Then, the support to simulate aggregate systems inside it has been designed. 
During this phase the focus has been on producing a platform that allows to move the simulated system to a real deployment with a low effort.
Finally, two different case studies have been illustrated to verify all the improvements and extensions introduced.
A demo of the first case study was showed during the Day of Science in Flanders to introduce the LoRaWAN technology receiving a lot of attention.
The second case study is actually evaluated only in qualitative way, but further evaluations of quantitative nature are under investigation and they will be the subject of a future publication.
At the end it is possible to say that the achieved platform enables the simulation of a aggregate system over a LoRaWAN network, and allows to move it in a real deploy in a simple way. 
To do so only two activities are necessary:
\begin{enumerate}
    \item change the time concept used from the Protelis nodes, from the simulator time to the time of the real device that hosts the node;
    \item change the MQTT client implementation (if it is used a mock one) and the MQTT broker address to subscribe to receive the mote packets.
\end{enumerate}
 So the entire aggregate system developed for the simulation (Protelis backend and program) does not require any changing.

\section{Future work}
Several improvements and interesting topics in different areas are available starting from the work illustrated in this thesis.

One area is to further improve the DingNet simulator. 
For example, it is possible to refine the model by simulating the loss of synchronism among the devices clock. 
It is possible to do so with different abstraction levels and required effort; starting with the application of a simple jitter to the clocks, arriving to introduce a stochastic model like a Markov chain.

Another area concerns the case study exposed in \cref{sec:case-staudyAC}. 
Here it is possible to evolve the program applying the techniques of machine learning to predict periods with high pollution and take countermeasures in advance. 
For example, it is possible to train a neural network that predicts the pollution level of the following days, considering the actual situation and the weather forecast.

Finally, an interesting topic deals with an extension of the thesis subject.
This thesis is focused on joining the aggregate computing with the communication technology LoRaWAN, and provide a platform to simulate this kind of systems.
But real complex pervasive scenarios do not use usually only one communication technology.
So, to support aggregate applications in real use case scenarios, a middleware is needed to fill the network abstraction gap introduced by the paradigm.
Consequently, the future work is to find an existing middleware that can fill this abstraction gap complying with the aggregate computing requirements. For example, Sentilo could be an interesting starting solution.