Recent technological developments led to increased computational and networking capabilities of everyday objects. 
This situation resulted in an increase in number of devices embedded in cyber-physical systems.
In order to simplify the design and management of pervasive and heterogeneous systems like these, there is need for new high-level paradigms able to capture concerns like heterogeneity and location of the devices.
Aggregate computing is one of these.
It proposes to describe the global behaviour of a system by managing global spatio-temporal data structures, and abstracting details of its physical network, as topology and communication technology.
Verifying the behaviour of a complex pervasive system in a real scenario is generally expensive, complicated, and not always possible in practice.
A partial solution to the problem is testing this kind of systems using simulation.
Even if those simulations execute some model of a system and that such is not the system itself, they can still provide realistic insight on the system behaviour and performance. % minuto 4.13
In the Internet-of-Things context, an emergent enabling communication technology for situated devices is LoRaWAN.
LoRaWAN is a network protocol that allows long range communication and low energy consumption, at the cost of limited data rate.
There are currently no platforms for aggregated languages that support them execution over LoRaWAN networks.
There are also currently no simulators that support real simulation of aggregate system over LoRaWAN networks, but there are simulators that support simulation of aggregate applications or LoRaWAN networks.
The contribution of this thesis is to provide a platform that supports the LoRaWAN abstractions as backend of an aggregate computing system, and join it to the existing DingNet simulator achieving a platform that allows the simulation of aggregate applications over realistic LoRaWAN networks. 
% al momento per i linuaggi aggregati non ci sono piattaforme che  supportino l'esecuzione su reti lorawan, 
% e ancor di più non ci sono nemmeno dei simulatori che abbiano del realismo che supportino sia reti lorawan che programmazione aggreggata
% esistono però sia simulatori in grado sia sim sw aggregato sia reti lorawan
% IL contributo di questa tesi è fornire una piattaforma che supporti le astrazioni di lora come backend di un sistema aggregate computing e collogare questa piattaforma alla sistema esistente DingNet in modo tale da avere una piattaforma che consenta di simulazione di computazione aggregate con reti lorawan realistche.