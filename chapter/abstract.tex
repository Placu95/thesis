Recent technological developments led to boost computational and networking capabilities of everyday objects. 
This situation has resulted in an increase in the number of devices embedded in cyber-physical systems.
To design and govern pervasive and heterogeneous systems like these are necessaries new paradigms whit a higher abstraction level to reduce their complexity.
Aggregate computing is one of these; it is a programming paradigm that proposes to describe the global behaviour of a system by abstracting all the details of its physical network.
Verify the behaviour of a complex pervasive system in a real scenario is expensive, complex, and not always possible.
A solution, to partially test these systems, is with platforms that enables simulations of this kind of systems.

In the Internet-of-Things context, an emergent and enabling communication technology for situated devices is LoRaWAN.
It is a network protocol that allows long range communication and low energy consumption, with the drawback of limited data rate.

Considering all, the purpose of this thesis is to start filling the network abstraction gap of the aggregate paradigm beginning with supporting aggregate computing over LoRaWAN networks.
To do so two main activities were carried out: design the support for Protelis, which is a language for aggregate computing, over LoRaWAN, and provide a simulation platform for this kind of systems. 